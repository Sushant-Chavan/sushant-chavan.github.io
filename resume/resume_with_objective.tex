%%%%%%%%%%%%%%%%%
% This is an sample CV template created using altacv.cls
% (v1.1.5, 1 December 2018) written by LianTze Lim (liantze@gmail.com). Now compiles with pdfLaTeX, XeLaTeX and LuaLaTeX.
%
%% It may be distributed and/or modified under the
%% conditions of the LaTeX Project Public License, either version 1.3
%% of this license or (at your option) any later version.
%% The latest version of this license is in
%%    http://www.latex-project.org/lppl.txt
%% and version 1.3 or later is part of all distributions of LaTeX
%% version 2003/12/01 or later.
%%%%%%%%%%%%%%%%

%% If you need to pass whatever options to xcolor
\PassOptionsToPackage{dvipsnames}{xcolor}

%% If you are using \orcid or academicons
%% icons, make sure you have the academicons
%% option here, and compile with XeLaTeX
%% or LuaLaTeX.
% \documentclass[10pt,a4paper,academicons]{altacv}

%% Use the "normalphoto" option if you want a normal photo instead of cropped to a circle
% \documentclass[10pt,a4paper,normalphoto]{altacv}

\documentclass[10pt,a4paper,ragged2e]{altacv}

%% AltaCV uses the fontawesome and academicon fonts
%% and packages.
%% See texdoc.net/pkg/fontawecome and http://texdoc.net/pkg/academicons for full list of symbols. You MUST compile with XeLaTeX or LuaLaTeX if you want to use academicons.

% Change the page layout if you need to
\geometry{left=1cm,right=9.65cm,marginparwidth=7.75cm,marginparsep=1cm,top=1.15cm,bottom=1.15cm}

% Change the font if you want to, depending on whether
% you're using pdflatex or xelatex/lualatex
\ifxetexorluatex
  % If using xelatex or lualatex:
  \setmainfont{Carlito}
\else
  % If using pdflatex:
  \usepackage[utf8]{inputenc}
  \usepackage[T1]{fontenc}
  \usepackage[default]{lato}
\fi

% Change the colours if you want to
\definecolor{Mulberry}{HTML}{72243D}
\definecolor{SlateGrey}{HTML}{2E2E2E}
\definecolor{LightGrey}{HTML}{444444}
\colorlet{heading}{Sepia}
\colorlet{accent}{SlateGrey}
\colorlet{emphasis}{Mulberry}
\colorlet{body}{LightGrey}

% \usepackage[none]{hyphenat}
\usepackage{hyperref}
\hypersetup{
    %bookmarks=true,         % show bookmarks bar?
    unicode=false,          % non-Latin characters in Acrobat’s bookmarks
    pdftoolbar=true,        % show Acrobat’s toolbar?
    pdfmenubar=true,        % show Acrobat’s menu?
    pdffitwindow=false,     % window fit to page when opened
    pdfstartview={FitH},    % fits the width of the page to the window
    pdftitle={Sushant Vijay Chavan},    % title
    pdfauthor={Sushant Vijay Chavan},     % author
    pdfsubject={Resume},   % subject of the document
    pdfcreator={Sushant Vijay Chavan},   % creator of the document
    pdfproducer={Sushant Vijay Chavan}, % producer of the document
    pdfkeywords={Resume, robotics, navigation}, % list of keywords
    pdfnewwindow=true,      % links in new PDF window
    colorlinks=true,       % false: boxed links; true: colored links
    linkcolor=body,          % color of internal links (change box color with linkbordercolor)
    citecolor=body,        % color of links to bibliography
    filecolor=body,      % color of file links
    urlcolor=body           % color of external links
}

% Change the bullets for itemize and rating marker
% for \cvskill if you want to
\renewcommand{\itemmarker}{{\small\textbullet}}
\renewcommand{\ratingmarker}{\faCircle}

%% sample.bib contains your publications
\addbibresource{sample.bib}

\begin{document}
\name{Sushant Vijay Chavan}
\tagline{
    % Graduate student at Hochschule Bonn-Rhein-Sieg
}
\photo{3cm}{Profile_circle.jpg}
\personalinfo{%
  % Not all of these are required!
  % You can add your own with \printinfo{symbol}{detail}
    % \begin{tabular}{ll}
    %     \phone{+49-163-9809177} & \location{Bonn, Germany} \\
    %     \email{sushant.vijay.chavan@gmail.com} & \homepage{sushant-chavan.github.io} \\
    %     \github{github.com/Sushant-Chavan} & \linkedin{linkedin.com/in/sushantvchavan}
    % \end{tabular}

%   \mailaddress{Grantham-Allee 21, 53757 Sankt Augustin, Germany}\\\vspace{0.1cm}
  \hspace{0.05cm}\location{\hspace{0.05cm}Bonn, Germany (Open to relocation)}
  \homepage{https://sushant-chavan.github.io}\\\medskip
  \email{sushant.vijay.chavan@gmail.com}\hspace{0.475cm}
  \linkedin{linkedin.com/in/sushantvchavan}\\\medskip
  \phone{\hspace{0.025cm}+49-163-9809177}\hspace{2.2cm}
  \github{github.com/Sushant-Chavan}
  %\twitter{@twitterhandle}
  %% You MUST add the academicons option to \documentclass, then compile with LuaLaTeX or XeLaTeX, if you want to use \orcid or other academicons commands.
  % \orcid{orcid.org/0000-0000-0000-0000} 
}

%% Make the header extend all the way to the right, if you want.
\begin{fullwidth}
% Include optional objective as an argument. Otherwise leave blank as: \makecvheader{}
\makecvheader{}
% \makecvheader{An experienced software professional seeking a full-time job opportunity in mobile robotics to apply the knowledge and skills gained throughout my academic and professional career.}
\end{fullwidth}

%% Depending on your tastes, you may want to make fonts of itemize environments slightly smaller
% \AtBeginEnvironment{itemize}{\small}

%% Provide the file name containing the sidebar contents as an optional parameter to \cvsection.
%% You can always just use \marginpar{...} if you do
%% not need to align the top of the contents to any
%% \cvsection title in the "main" bar.

% \cvsection[page1sidebar]{Career Objective}
% \begin{fullwidth}
% An experienced software professional seeking a full-time job opportunity in robotics to apply the knowledge and skills \\gained throughout my academic and professional career.
% \end{fullwidth}


\cvsection[page1sidebar]{Work Experience}

\cvevent{Student Research Assistant}{KELO Robotics GmbH}{Jul 2020 -- Present}{Sankt Augustin, Germany}
\begin{itemize}
    % \item Propose and use a novel semantic map representation, based on GeoJSON and OpenStreetMap representations
    \item Propose and use a novel indoor semantic map representation
    \item Design and implement single/multi robot simulation software using Gazebo and Unity3D (with ROS-Sharp) respectively
\end{itemize}

\divider

\cvevent{Student Research Assistant}{ROPOD Project, Hochschule Bonn-Rhein-Sieg}{Jun 2019 -- Apr 2020}{Sankt Augustin, Germany}
% Developed software for simulating a multi-robot fleet management system. Implemented continuous integration tests for evaluating the coordination and resource usage by the robot fleet.
\begin{itemize}
    \item Develop a flexible Gazebo-based simulator for testing high-level execution of tasks by a fleet of mobile robots
    % \item Setup and maintain continuous integration of software packages using Docker and GitLab CI
    \item Setup and maintain continuous integration of software packages
\end{itemize}

\divider

\cvevent{Senior Software Engineer}{Robert Bosch Engineering and Business Solutions Private Limited}{Sep 2013 -- Dec 2017}{Bangalore, India / Hildesheim, Germany}
% Fulfilled various roles as a C++ \& 3D computer graphics developer. I was responsible for the design and development of next-gen, enhanced 3D navigation map graphics features, that are optimized to run on low-end embedded in-vehicle infotainment systems.
\begin{itemize}
    \item OpenSceneGraph \& Navigation Data Standard (NDS) based 3D map visualization application for in-vehicle infotainment systems
    \item Implement highly optimized algorithms for rendering 3D building models on embedded devices with limited computing power
\end{itemize}

\cvsection{Relevant Projects}

\cvevent{Composable Ontology-based Indoor Semantic Map}{Master Thesis, Hochschule Bonn-Rhein-Sieg}{}{}
\begin{itemize}
    \item A novel framework to flexibly compose new maps from a prior collection of heterogeneous metric-semantic indoor maps
    \item Employ an ontological knowledge-base together with the composed maps to plan context-dependent indoor navigation plans
\end{itemize}


\divider

\cvevent{Simulator for modular KELO ROBILE mobile robots  \hfill \githubproject{https://github.com/kelo-robotics/robile_gazebo}}{KELO Robotics GmbH}{}{}
\begin{itemize}
    \item A ROS package to simulate arbitrary configurations of modular mobile robots in Gazebo with an option to activate true physics-based wheel motion, resulting in the motion of the platform
\end{itemize}

\divider

\cvevent{Comparative evaluation of experience-based path planners for multi robot Systems \hfill \githubproject{https://github.com/Sushant-Chavan/coordination_oru\#experience-based-planning}}{Hochschule Bonn-Rhein-Sieg}{}{}
\begin{itemize}
    \item Port and quantitatively evaluate the suitability of using single robot experience-based path planners for a fleet of logistic mobile robots, using a custom-developed benchmarking framework
\end{itemize}

% \cvevent{Development of RViz plugins for RoboCup teams}{b-it-bots, Hochschule Bonn-Rhein-Sieg}{Oct 2018 – Jan 2019}{}
% %\begin{itemize}
% %	\item We developed a \textit{Map Annotation Tool} that eases annotating robot navigation maps with goal poses, by allowing storage, retrieval, and editing of navigational goal markers.
% %	\item We developed a \textit{Action Client Panel} that eases the task of sending goals to an action server. The panel provides a GUI for selecting actions, setting their parameters, sending action goals and monitoring progress of action execution.
% %\end{itemize}
% \begin{itemize}
% \item  We developed two RViz plugins called \textit{Map Annotation Tool} and \textit{Action Client Panel}. The map annotation tool facilitates annotating navigation maps efficiently and flexibly. The action client panel provides a GUI for selecting actions, setting their parameters, sending action goals and monitoring the progress of action execution.
% \end{itemize}

%\medskip
%
%\cvsection{A Day of My Life}
%
%% Adapted from @Jake's answer from http://tex.stackexchange.com/a/82729/226
%% \wheelchart{outer radius}{inner radius}{
%% comma-separated list of value/text width/color/detail}
%\wheelchart{1.5cm}{0.5cm}{%
%  6/8em/accent!30/{Sleep,\\beautiful sleep},
%  3/8em/accent!40/Hopeful novelist by night,
%  8/8em/accent!60/Daytime job,
%  2/10em/accent/Sports and relaxation,
%  5/6em/accent!20/Spending time with family
%}

%\clearpage
%\cvsection[page2sidebar]{Publications}
%
%\nocite{*}

%\printbibliography[heading=pubtype,title={\printinfo{\faBook}{Books}},type=book]
%
%\divider
%
%\printbibliography[heading=pubtype,title={\printinfo{\faFileTextO}{Journal Articles}},type=article]
%
%\divider
%
%\printbibliography[heading=pubtype,title={\printinfo{\faGroup}{Conference Proceedings}},type=inproceedings]

%% If the NEXT page doesn't start with a \cvsection but you'd
%% still like to add a sidebar, then use this command on THIS
%% page to add it. The optional argument lets you pull up the
%% sidebar a bit so that it looks aligned with the top of the
%% main column.
% \addnextpagesidebar[-1ex]{page3sidebar}


\end{document}
